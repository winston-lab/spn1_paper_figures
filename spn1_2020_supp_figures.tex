\documentclass[8pt]{extarticle}
\setlength\parindent{0pt}
\linespread{1}
\pagestyle{empty}

% \usepackage[letterpaper, left=18.95mm, right=18.95mm, top=24.7mm, bottom=24.7mm, showframe]{geometry}
\usepackage[letterpaper, left=18.95mm, right=18.95mm, top=24.7mm, bottom=24.7mm]{geometry}

\usepackage{amsmath}
\usepackage{mathspec}
\setmainfont{FreeSans}[
    Path=fonts/,
    BoldFont=FreeSansBold,
    ItalicFont=FreeSansOblique,
    BoldItalicFont=FreeSansBoldOblique
]
\setmathsfont(Digits,Greek)[Path=fonts/]{FreeSans}
\setmathrm[Path=fonts/]{FreeSans}

\usepackage{graphicx}
\usepackage{textcomp}
\usepackage{enumitem}

\begin{document}

% \textbf{\large Supplementary Figure S1}

{\includegraphics[width=17.8cm]{figures/figure_S1_spn1_depletion_supplemental.pdf}\par}

\vspace{2em}
\textbf{Supplementary Figure S1.} Effects of Spn1 depletion.
(\textbf{A}) Viability of a wild-type strain, a strain expressing TIR1 but lacking the degron tag on Spn1 (Spn1-noAID), and the Spn1 degron strain (Spn1-AID) after 90 minutes of treatment with DMSO or 25 $\mu$M IAA, quantified as colony-forming units per OD$_{600}$ unit of culture plated.
Error bars indicate the mean $\pm$ standard deviation of the replicates shown.
(\textbf{B}) Scatterplots of Spn1 enrichment versus Rpb1 enrichment for 5,091 verified coding genes in non-depleted and Spn1-depleted conditions.
Pearson correlation coefficients are indicated.
Enrichment values are the relative log$_2$ enrichment of IP over input.
(\textbf{C}) Scatterplot showing Rpb1-normalized Spn1 enrichment versus Rpb1 enrichment for the same genes as in (B) in the non-depleted condition.
Rpb1 enrichment values are the relative log$_2$ enrichment of IP over input.
The Pearson correlation coefficient is indicated, and a linear regression is overlaid.

\newpage

% \textbf{\large Supplementary Figure 2}

{\includegraphics[width=17.8cm]{figures/figure_S2_rnaseq_rpb1_supplemental.pdf}\par}

\vspace{2em}
\textbf{Supplementary Figure S2.} RNA-seq and RNAPII ChIP-seq.
(\textbf{A}) Scatterplot comparing the results of two RNA-seq differential expression analyses for 5,091 verified coding genes.
X-axis values show the fold-change estimated from comparison of the Spn1 degron strain treated with IAA (Spn1-depleted) versus DMSO (non-depleted), the same comparison shown in Figure 2A.
Y-axis values show the fold-change attributable to the effects of Spn1 depletion, as extracted from a multi-factor analysis incorporating RNA-seq data from additional control strains to be able to separate effects due to Spn1 depletion from effects due to presence of the degron tag on Spn1, expression of degron system component TIR1, or treatment of yeast cells with IAA.
The line y=x is overlaid for comparison.
Results of the simple comparison between Spn1-depleted and non-depleted conditions are used throughout this study for consistency with ChIP-seq analyses for which the multi-factor analysis was not performed.
(\textbf{B}) Scatterplot comparing changes in transcript abundance upon Spn1 depletion as measured by RT-qPCR and RNA-seq. The line y=x is drawn for comparison, and error bars indicate 95\% confidence intervals.
(\textbf{C}) Running GSEA enrichment score for genes up- and downregulated during the yeast environmental stress response (ESR) (43) along genes ranked by change in transcript abundance upon Spn1 depletion as measured by RNA-seq.
The ranks of ESR up- and downregulated genes are marked along the top and bottom of the plot, respectively, and Benjamini-Hochberg adjusted enrichment p-values are indicated.
(\textbf{D}) Scatterplot comparing changes in transcript abundance upon Spn1 depletion as measured by RNA-seq versus changes in transcript abundance expected from a slow growth signature common to many environmental and genetic perturbations as previously reported in (44).
(\textbf{E}) Spike-in normalized RNA-seq coverage over four genes with antisense transcripts that are differentially upregulated upon Spn1 depletion.
Sense and antisense strand signal are plotted above and below the x-axis, respectively.
Unlabeled horizontal lines indicate the boundaries of transcripts called by StringTie.
(\textbf{F}) Barplots showing total levels of Rpb1, Rpb1-Ser5-P, and Rpb1-Ser2-P on chromatin in non-depleted and Spn1-depleted conditions, estimated from ChIP-seq spike-in normalization factors (see Methods).
The values and error bars for each condition indicate the mean $\pm$ standard deviation of the replicates shown.
Note that we observed batch-to-batch variability in Rpb1 levels after Spn1 depletion, with Rpb1 levels decreasing for the two replicates generated from the same chromatin preparations used to ChIP Rpb1-Ser5-P and Rpb1-Ser2-P.

\newpage

% \textbf{\large Supplementary Figure 3}

{\includegraphics[width=17.8cm]{figures/figure_S3_set2_spt6_supplemental.pdf}\par}

\vspace{2em}
\textbf{Supplementary Figure S3.} Spt6 and Set2 ChIP-seq.
(\textbf{A}) Scatterplots showing Spt6 or Set2 enrichment versus Rpb1 enrichment for 5,091 verified coding genes in non-depleted and Spn1-depleted conditions.
Enrichment values are the relative log$_2$ enrichment of IP over input.
Pearson correlation coefficients are indicated.
(\textbf{B}) Scatterplots showing Rpb1-normalized enrichment of Spt6 or Set2 versus Rpb1 enrichment for the same genes as in (A) in non-depleted and Spn1-depleted conditions.
Pearson correlation coefficients are indicated.
(\textbf{C}) Representative western blots showing the levels of Spn1, Rpb1, Spt6, and Set2 in the wild type, Spn1-noAID, and Spn1-AID strains treated with DMSO or 25 $\mu$M IAA.
The faint band co-migrating with native Spn1 in the Spn1-AID lanes of the Spn1 blot is non-specific, as it was also present in a \textit{spn1$\Delta$} strain (data not shown).
Pgk1 was used as a loading control.

\newpage

% \textbf{\large Supplementary Figure 4}

{\includegraphics[width=17.8cm]{figures/figure_S4_h3_supplemental.pdf}\par}

\vspace{2em}
\textbf{Supplementary Figure S4.} Histone H3 ChIP-seq.
(\textbf{A}) Average H3 ChIP enrichment for 3,087 non-overlapping, verified coding genes aligned by TSS in non-depleted and Spn1-depleted conditions.
The solid line and shading are the median and interquartile range of the mean enrichment over four replicates.
(\textbf{B}) Scatterplot showing change in H3 enrichment upon Spn1 depletion over the first 500 bp downstream of the TSS versus non-depleted Rpb1 enrichment over the entire length of 4,924 verified coding genes at least 500 bp long.
Rpb1 enrichment values are the relative log$_2$ enrichment of IP over input.
Genes with significant decreases in H3 enrichment over their first 500 bp are colored red, and a cubic regression spline is overlaid.
(\textbf{C}) Scatterplots comparing Spn1-dependent changes in H3 enrichment over the entire gene to changes in Rpb1 enrichment, Spt6 enrichment, and Rpb1-normalized Spt6 enrichment.
Values are shown for 5,091 verified coding genes, with genes for which H3 enrichment is significantly reduced upon Spn1 depletion colored red.
A linear regression for each group is overlaid, with Pearson correlation coefficients shown in the bottom right.
(\textbf{D}) Average H3, Rpb1, Spt6, and Rpb1-normalized Spt6 ChIP enrichment in non-depleted and Spn1-depleted conditions for the 77 genes with significantly reduced H3 enrichment upon Spn1 depletion and for 80 genes without significantly reduced H3 enrichment matched to the first group by expression level and transcript length.
The solid line and shading are the median and interquartile range over the genes in each group.

\newpage

% \textbf{\large Supplementary Figure 5}

{\includegraphics[width=17.8cm]{figures/figure_S5_h3_mods_supplemental.pdf}\par}

\vspace{2em}
\textbf{Supplementary Figure S5.} Histone H3 modifications ChIP-seq.
(\textbf{A}) Representative western blots showing the levels of H3K36me2, H3K36me3, and H3 in the wild type, Spn1-noAID, and Spn1-AID strains treated with DMSO or 25 $\mu$M IAA.
The same Spn1 and Pgk1 panels used in Supplementary Figure S3C are shown here for reference, as they were generated as part of the same experiment.
(\textbf{B}) Barplots showing total levels of H3K36me2, H3K36me3, H3K4me3, and H3 on chromatin in non-depleted and Spn1-depleted conditions, estimated from ChIP-seq spike-in normalization factors (see Methods).
The values and error bars for each condition indicate the mean $\pm$ standard deviation of the replicates shown.
(\textbf{C}) Average H3K36me2, H3K36me3, and H3K4me3 ChIP enrichment in non-depleted and Spn1-depleted conditions for 3,087 non-overlapping, verified coding genes aligned by TSS.
The solid line and shading are the median and interquartile range of the mean enrichment over two replicates.
(\textbf{D}) Average H3-normalized H3K36me2, H3K36me3, and H3K4me3 ChIP enrichment in non-depleted and Spn1-depleted conditions for 3,087 non-overlapping, verified coding genes separated into tertiles of expression by their non-depleted Rpb1 ChIP enrichment.
The solid line and shading are the median and interquartile range of the mean ratio over two replicates.

\newpage

% \textbf{\large Supplementary Figure 6}

% {\includegraphics[width=17.8cm]{figure_S6_chd1.pdf}\par}

% \newpage

% \textbf{\large Supplementary Figure 6}

{\includegraphics[width=17.8cm]{figures/figure_S7_splicing_supplemental.pdf}\par}

\vspace{2em}
\textbf{Supplementary Figure S6.} Effects of Spn1 depletion at intron-containing ribosomal protein genes.
(\textbf{A}) Genomic data visualization for ribosomal protein genes with and without introns.
The solid line and shading for each assay are the median and interquartile range over the genes in each group.
The same sense strand RNA-seq coverage panel used in Figure 7C is shown here for reference.
(\textbf{B}) Heatmaps of change upon Spn1 depletion for sense strand RNA-seq signal and H3-normalized H3K36me2 ChIP enrichment over 89 ribosomal protein (RP) genes with introns and 153 non-RP genes with introns.
Genes are aligned by TSS and ordered by distance from the TSS to the midpoint of the first intron.
The 5$^\prime$ and 3$^\prime$ splice sites of each intron are marked with red and orange dots, respectively.

\newpage

\end{document}
